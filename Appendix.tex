\begin{lstlisting}[caption=ElementDictionary, label=listing2]
 /**
	 * The dictionary indexes the elements vertically and horizontally for fast
	 * retrieval.
	 * 
	 * @author ryan
	 * 
	 */
	class ElementDictionary {
		private HashMap<Integer, LinkedList<Element>> v_map, h_map;
		private ArrayList<Element> inc, exc;
		private boolean hasExcludeChanged;

		ElementDictionary(ArrayList<Element> exclude) {
			v_map = new HashMap<Integer, LinkedList<Element>>();
			h_map = new HashMap<Integer, LinkedList<Element>>();
			this.inc = new ArrayList<Element>();
			this.exc = exclude;
			hasExcludeChanged = true;
		}

		private void build() {
			v_map.clear();
			h_map.clear();
			ArrayList<Element> all = getElements();
			all.removeAll(exc);
			addAllToInc(all);
			for (Element elt : all) {
				subAdd(elt, elt.getX(), elt.getY(), elt.getWidth(),
						elt.getHeight());
				inc.add(elt);
			}
			hasExcludeChanged = false;
		}

		public void addToExc(Element elt) {
			if (!inc.contains(elt) && !exc.contains(elt)) {
				exc.add(elt);
			}
		}

		public void addAllToExc(List<Element> elts) {
			for (Element elt : elts) {
				addToExc(elt);
			}
		}

		public void addToInc(Element elt) {
			if (!inc.contains(elt) && !exc.contains(elt)) {
				subAdd(elt, elt.getX(), elt.getY(), elt.getWidth(),
						elt.getHeight());
				inc.add(elt);
			}
		}

		public void addAllToInc(List<Element> elts) {
			for (Element elt : elts) {
				addToInc(elt);
			}
		}

		public void include(Element elt) {
			assert (!inc.contains(elt) && exc.contains(elt))
					|| (inc.contains(elt) && !exc.contains(elt));
			if (exc.contains(elt)) {
				exc.remove(elt);
				subAdd(elt, elt.getX(), elt.getY(), elt.getWidth(),
						elt.getHeight());
				inc.add(elt);
			}
		}

		public void includeAll(List<Element> elts) {
			for (Element elt : elts) {
				include(elt);
			}
		}

		public void includeAll() {
			includeAll((ArrayList<Element>) exc.clone());
			exc.clear();
		}

		public void exclude(Element elt) {
			assert (!inc.contains(elt) && exc.contains(elt))
					|| (inc.contains(elt) && !exc.contains(elt));
			if (inc.contains(elt)) {
				inc.remove(elt);
				subRemove(elt, elt.getX(), elt.getY(), elt.getWidth(),
						elt.getHeight());
				exc.add(elt);
			}
		}

		public void excludeAll(List<Element> elts) {
			for (Element elt : elts) {
				exclude(elt);
			}
		}

		public void excludeAll() {
			excludeAll((ArrayList<Element>) inc.clone());
			inc.clear();
		}

		public void remove(Element elt) {
			if (inc.remove(elt)) {
				subRemove(elt, elt.getX(), elt.getY(), elt.getWidth(),
						elt.getHeight());
				assert !exc.contains(elt);
			} else if (exc.remove(elt)) {
				// Do nothing.
			}
		}

		public void removeAll(List<Element> elts) {
			for (Element elt : elts) {
				remove(elt);
			}
		}
		
		public void rotate(Element elt){
			if (inc.contains(elt)){
				subRemove(elt, elt.getX(), elt.getY(), elt.getOldWidth(),
						elt.getOldHeight());
				subAdd(elt, elt.getX(), elt.getY(), elt.getWidth(),
						elt.getHeight());
			}
		}

		public void scale(Element elt) {
			if (inc.contains(elt)) {
				subRemove(elt, elt.getX(), elt.getY(), elt.getOldWidth(),
						elt.getOldHeight());
				subAdd(elt, elt.getX(), elt.getY(), elt.getWidth(),
						elt.getHeight());
			}
		}

		public void move(Element elt) {
			if (inc.contains(elt)) {
				subRemove(elt, elt.getOldX(), elt.getOldY(), elt.getWidth(),
						elt.getHeight());
				subAdd(elt, elt.getX(), elt.getY(), elt.getWidth(),
						elt.getHeight());
			}
		}

		public void moveAll(List<Element> elts) {
			for (Element elt : elts) {
				move(elt);
			}
		}

		private void subAdd(Element elt, int x, int y, int w, int h) {
			for (int i = x; i <= x + w; i++) {
				LinkedList<Element> list = v_map.get(i);
				if (list == null) {
					list = new LinkedList<Element>();
					v_map.put(i, list);
				}
				list.add(elt);
			}
			for (int j = y; j <= y + h; j++) {
				LinkedList<Element> list = h_map.get(j);
				if (list == null) {
					list = new LinkedList<Element>();
					h_map.put(j, list);
				}
				list.add(elt);
			}
		}

		private void subRemove(Element elt, int x, int y, int w, int h) {
			for (int i = x; i <= x + w; i++) {
				LinkedList<Element> list = v_map.get(i);
				assert list != null;
				list.remove(elt);
			}

			for (int j = y; j <= y + h; j++) {
				LinkedList<Element> list = h_map.get(j);
				assert list != null;
				list.remove(elt);
			}
		}

		// public void setExcludeChanged() {
		// hasExcludeChanged = true;
		// }
		//
		// public void setExclude(ArrayList<Element> exc) {
		// includeAll(this.exc);
		// this.inc.addAll(exc);
		// this.exc = exc;
		// hasExcludeChanged = true;
		// }

		// public void update(List<Element> moved_elts) {
		//
		// }

		public LinkedList<Element> getElementsAtRow(int y) {
			if (hasExcludeChanged) {
				build();
				hasExcludeChanged = false;
			}
			return h_map.get(y);
		}

		public LinkedList<Element> getElementsAtColumn(int x) {
			if (hasExcludeChanged) {
				build();
				hasExcludeChanged = false;
			}
			return v_map.get(x);
		}

		/**
		 * 
		 * @param elt
		 * @return true - if 'elt' overlaps any Element in the dictionary
		 *         (includeing the case 'elt' is in the dictionary and is not in
		 *         exclude).
		 */
		public boolean overlapping(Element elt) {
			return overlapping(elt.getX(), elt.getY(), elt.getWidth(),
					elt.getHeight());
		}

		/**
		 * 
		 * @param x
		 * @param y
		 * @param w
		 * @param h
		 * @return
		 */
		public boolean overlapping(int x, int y, int w, int h) {
			if (hasExcludeChanged) {
				build();
				hasExcludeChanged = false;
			}
			if (w <= 1) {
				boolean b1 = true, b2 = true;
				if (overlappingSub(x, y, h, true)) {
					b1 = overlappingSub(x + 1, y, h, true);
					if (w == 0)
						b2 = overlappingSub(x - 1, y, h, true);
					if (b1 && b2)
						return true;
				}
			}
			for (int i = x + 1; i < x + w; i++) {
				if (overlappingSub(i, y, h, true)) {
					return true;
				}
			}
			if (h <= 1) {
				boolean b1 = true, b2 = true;
				if (overlappingSub(y, x, w, false)) {
					b1 = overlappingSub(y + 1, x, w, false);
					if (h == 0)
						b2 = overlappingSub(y - 1, x, w, false);
					if (b1 && b2)
						return true;
				}
			}
			for (int j = y + 1; j < y + h; j++) {
				if (overlappingSub(j, x, w, false)) {
					return true;
				}
			}
			return false;
		}

		/**
		 * Decide if the line formed by i, base and span overlaps any Element.
		 * If isVertical == true, then the line is (i, base) (i, base+span); if
		 * isVertical == false, then the line is (base, i), (base+span, i)
		 * 
		 * @param i
		 *            - The coordinate that the end points of the line share.
		 * @param base
		 * @param span
		 * @param isVertical
		 * @return
		 */
		private boolean overlappingSub(int i, int base, int span,
				boolean isVertical) {
			if (isVertical) {
				LinkedList<Element> list = v_map.get(i);
				if (list != null) {
					for (Element tgt : list) {
						int ty1 = tgt.getY();
						int ty2 = ty1 + tgt.getHeight();
						int ey1 = base;
						int ey2 = ey1 + span;
						if (Decider.overlap(ty1, ty2, ey1, ey2, false)) {
							return true;
						}
					}
				}
			} else {
				LinkedList<Element> list = dic_exc.getElementsAtRow(i);
				if (list != null) {
					for (Element tgt : list) {
						int tx1 = tgt.getX();
						int tx2 = tx1 + tgt.getWidth();
						int ex1 = base;
						int ex2 = ex1 + span;
						if (Decider.overlap(tx1, tx2, ex1, ex2, false)) {
							return true;
						}
					}
				}
			}
			return false;
		}
	}
\end{lstlisting}
\begin{lstlisting}[caption=EdgeManager, label=listing3]
 public class EdgeManager {
	private WireManager wm;
	private ElementManager em;
	private HashMap<Integer, ArrayList<WireEdge>> h_edges, v_edges;
	private ArrayList<Edge> overlap;
	private boolean needToBuild, needToRenewOverlap;

	public EdgeManager() {
		h_edges = new HashMap<Integer, ArrayList<WireEdge>>();
		v_edges = new HashMap<Integer, ArrayList<WireEdge>>();
		// h_ps = new HashMap<Integer, ArrayList<RoutingPoint>>();
		// v_ps = new HashMap<Integer, ArrayList<RoutingPoint>>();
		overlap = new ArrayList<Edge>();
		needToBuild = true;
		needToRenewOverlap = true;
	}

	public void init() {
		this.wm = CircuitManager.getInstance().getWireManager();
		this.em = CircuitManager.getInstance().getElementManager();
	}
	
	public void clear(){
		h_edges = new HashMap<Integer, ArrayList<WireEdge>>();
		v_edges = new HashMap<Integer, ArrayList<WireEdge>>();
		// h_ps = new HashMap<Integer, ArrayList<RoutingPoint>>();
		// v_ps = new HashMap<Integer, ArrayList<RoutingPoint>>();
		overlap = new ArrayList<Edge>();
		needToBuild = true;
		needToRenewOverlap = true;
	}

	protected void build() {
		h_edges.clear();
		v_edges.clear();
		for (Wire wire : wm.getWires()) {
			subAddWire(wire);
		}
		needToBuild = false;
		needToRenewOverlap = true;
	}

	private void subAddWire(Wire wire) {
		for (WireEdge edge : wire.getRoutingEdges()) {
			if (edge.isVertical()) {
				ArrayList<WireEdge> list = v_edges.get(edge.getP1().getX());
				if (list == null) {
					list = new ArrayList<WireEdge>();
					v_edges.put(edge.getP1().getX(), list);
				}
				v_edges.get(edge.getP1().getX()).add(edge);
			} else {
				ArrayList<WireEdge> list = h_edges.get(edge.getP1().getY());
				if (list == null) {
					list = new ArrayList<WireEdge>();
					h_edges.put(edge.getP1().getY(), list);
				}
				h_edges.get(edge.getP1().getY()).add(edge);
			}
		}
	}

	/**
	 * 
	 * @return An ArrayList of Edges representing the places of overlapping.
	 *         Note the Edges are not RoutingEdges.
	 */
	public ArrayList<Edge> getOverlapping() {
		if (needToRenewOverlap) {
			overlap.clear();
			build();
			// Overlapping between Wires
			for (ArrayList<WireEdge> list : v_edges.values()) {
				for (int i = 0; i < list.size(); i++) {
					for (int j = i + 1; j < list.size(); j++) {
						WireEdge e1 = list.get(i);
						WireEdge e2 = list.get(j);
						Edge overlap_edge = WireEdge.getOverlap(e1, e2);
						if (overlap_edge != null) {
							overlap.add(overlap_edge);
						}
					}

				}
			}
			for (ArrayList<WireEdge> list : h_edges.values()) {
				for (int i = 0; i < list.size(); i++) {
					for (int j = i + 1; j < list.size(); j++) {
						WireEdge e1 = list.get(i);
						WireEdge e2 = list.get(j);
						Edge overlap_edge = WireEdge.getOverlap(e1, e2);
						if (overlap_edge != null) {
							overlap.add(overlap_edge);
						}
					}
				}
			}
			ElementDictionary dic = em.getElementDictionary();
			// Overlapping between Wires and Elements
			for (Integer key : v_edges.keySet()) {
				for (WireEdge edge : v_edges.get(key)) {
					LinkedList<Element> list = dic.getElementsAtColumn(key);
					if (list != null) {
						for (Element elt : list) {
							Edge overlap_edge = WireEdge.getOverlap(edge, elt);
							if (overlap_edge != null) {
								overlap.add(overlap_edge);
							}
						}
					}
				}
			}
			for (Integer key : h_edges.keySet()) {
				for (WireEdge edge : h_edges.get(key)) {
					LinkedList<Element> list = dic.getElementsAtRow(key);
					if (list != null) {
						for (Element elt : list) {
							Edge overlap_edge = WireEdge.getOverlap(edge, elt);
							if (overlap_edge != null) {
								overlap.add(overlap_edge);
							}
						}
					}
				}
			}
			needToRenewOverlap = false;
		}
		return overlap;
	}

	public WireEdge getEdgeAt(int x, int y) {
		if (needToBuild) {
			build();
		}
		ArrayList<WireEdge> list = v_edges.get(x);
		if (list != null) {
			for (WireEdge edge : list) {
				int min_y = Math.min(edge.getP1().getY(), edge.getP2().getY());
				int max_y = Math.max(edge.getP1().getY(), edge.getP2().getY());
				if (min_y <= y && y < max_y) {
					return edge;
				}
			}
		}
		list = h_edges.get(y);
		if (list != null) {
			for (WireEdge edge : list) {
				int min_x = Math.min(edge.getP1().getX(), edge.getP2().getX());
				int max_x = Math.max(edge.getP1().getX(), edge.getP2().getX());
				if (min_x <= x && x < max_x) {
					return edge;
				}
			}
		}
		return null;
	}

	protected void update() {
		needToBuild = true;
		needToRenewOverlap = true;
	}

	protected void needToBuild() {
		needToBuild = true;
	}

	protected void needToRenewOverlap() {
		needToRenewOverlap = true;
	}
}

\end{lstlisting}

\begin{table}
 \begin{tabular}{p{6cm} | p{8cm}} 
  Command Name & Parameters \\
 \hline
  CREATE\_GATE & x:int - the x-coordinate of the new gate\\
	      & y:int - the y-coordinate of the new gate \\
 \hline
  CREATE_FANOUT & wire:Wire - the wire on which the fanout will be added \\
		& index:int - the index of the wire edge that the fanout will be added to \\
		& x:int - the x-coordinate of the fanout \\
		& y:int - the y-coordinate of the fanout \\
 \hline
  CREATE_WIRE_EDGE & wire:Wire - the wire on which the new edge will be created \\
		  & index:int - the index of the wire edge on which the new edge will be added to. \\
		  & x1:int - the x-coordinate of the first point \\
		  & y1:int - the y-coordinate of the first point \\
		  & x2:int - the x-coordinate of the second point \\
		  & y2:int - the y-coordinate of the second point \\
 \hline
  DELETE & \\
 \hline
  MOVE & dx:int - the x displacement \\
      & dy:int - the y displacement \\
 \hline
  ROTATE & isClockWise:boolean - if the rotation is by 90 degrees clockwise \\
 \hline
  CHANGE_NUMBER_OF_INPORT & num_ips:int - the new number of inports \\
 \hline
  CHANGE_NUMBER_OF_OUTPORT & num_ops:int - the new number of outports \\
 \hline
  CHANGE_MAX_DELAY & max_delay:long - the new value of maximum delay \\
 \hline
  CHANGE_MIN_DELAY & min_delay:long - the new value of minimum delay \\
 \hline
  SELECT_WIRE & wire:Wire - the wire to be selected \\
 \hline
  SELECT_WIRE_EDGE & edge:WireEdge - the WireEdge to be selected \\
 \hline
  SELECT_ELEMENT & elt:Element - the Element to be selected \\
 \hline
  SELECT_MULTI_ELEMENT & elt:Element - the Element to be taken xor with the selected Elements \\
 \hline
  SELECT_GROUP_ELEMENT & elts:List<Element> - the list of Elements to be selected.  Previous selection will be deselected. \\
 \hline
  SELECT_GROUP_MULTI_ELEMENT & elts:List<Element> - the list of elements to be added to the selection. \\
 \hline
  DESELECT & \\
 \hline
  UNDO & \\
 \hline
  REDO & \\
 \hline
  COPY & \\
 \hline
  PASTE & \\

 \end{tabular}
\caption{A list of command names and parameters for the editor} \label{tab:table1}
\end{table}

\begin{lstlisting}[caption=ElementDictionary, label=listing7]
public abstract class AbstractScript {
	private LinkedList<DelayValuePair> history = new LinkedList<DelayValuePair>();

	abstract DelayValuePair getNextPair();

	/**
	 * Restoring all children Loops to their initial value. This is for starting
	 * a new outer Loop.
	 */
	abstract public void restore();

	public DelayValuePair getNextPairLogging() {
		DelayValuePair temp = getNextPair();
		if (temp != null) {
			history.addLast(temp);
		}
		return temp;
	}
	
	public void restoreCleanseHistory(){
		this.restore();
		history.clear();
	}
	
	public String historyToString(){
		StringBuffer buffer = new StringBuffer();
		for (DelayValuePair pair:history){
			buffer.append(pair.toString());
		}
		return buffer.toString();
	}

	// ====================================================
	// AbstractScriptParser
	// ====================================================

	public static AbstractScript parse(String string) {
		if (isValid(string)) {
			return subParse(string);
		} else {
			return null;
		}
	}

	public static boolean isValid(String string) {
		string = string.trim();
		int i = isFirstOrSecondType(string);
		int j = isThirdorFourthType(string);
		System.out.println("i==" + i + ", j==" + j);
		System.out
				.println("\""
						+ string
						+ "\" is "
						+ (i > 0 ? "type \"loop n (S) S\""
								: (i == -1 ? "type \"loop n (S)\""
										: (j > 0 ? "type \"<long, LogicValue> S\""
												: (j == -1 ? "type \"<long, LogicValue>\""
														: "Invalid")))));
		System.out.println();
		if (((i > 0 || i == -1) && j == 0) || (i == 0 && (j > 0 || j == -1))) {
			if (i > 0) { // Type "loop n (S) S"
				return isValid(string.substring(0, i))
						&& isValid(string.substring(i));
			} else if (i == -1) { // Type "loop n (S)"

				Pattern num_loop_pattern = Pattern
						.compile("loop\\s+[0-9]+\\s*");
				Matcher num_loop_matcher = num_loop_pattern.matcher(string);
				num_loop_matcher.find();
				int after_loop_num_index;
				try {
					after_loop_num_index = num_loop_matcher.end();
				} catch (IllegalStateException e) {
					return false;
				}

				String substring = string.substring(after_loop_num_index + 1,
						string.length() - 1);
				System.out.println("\"" + string + "\" has substring \""
						+ substring + "\"");
				return isValid(substring);
			} else if (j > 0) { // Type "<long, LogicValue> S"
				return isValid(string.substring(0, j))
						&& isValid(string.substring(j));
			} else { // Type "<long, LogicValue>"
				assert j == -1;
				return Pattern.matches("<\\s*[0-9]+\\s*,\\s*[01xX]\\s*>",
						string);
			}
		} else {
			return false;
		}
	};

	/**
	 * A recursive method employing a divide and conquer idea.
	 * 
	 * The input string has the following CFG: S = loop n (S) | loop n (S) S |
	 * <long, LogicValue> S | <long, LogicValue>
	 * 
	 * @param input
	 *            - of type {String, Map<String, AbstractScript>}
	 * @return
	 */
	private static AbstractScript subParse(String input) {

		String str = input.trim();

		Pattern single_pair_pattern = Pattern
				.compile("<\\s*[0-9]+\\s*,\\s*[01xX]\\s*>"); // Eg, %var1%
		Matcher single_pair_matcher = single_pair_pattern.matcher(str);
		if (single_pair_matcher.matches()) { // Type "<long, LogicValue>"
			// Split the input string into two parts
			String delims = ",\\s*";
			String[] tokens = str.split(delims);
			assert tokens.length == 2;

			// Match the delay
			Pattern number_pattern = Pattern.compile("[0-9]+");
			Matcher number_matcher = number_pattern.matcher(tokens[0]);
			number_matcher.find();
			String delay_string = number_matcher.group();
			long delay = Long.parseLong(delay_string);

			// Match the logic value
			String second_part = tokens[1];
			// Try to get the very first character of second_part
			String logic_string = second_part.substring(0, 1);
			LogicValue logic_value = getLogicValue(logic_string);
			DelayValuePair delay_value_pair = new DelayValuePair(delay,
					logic_value);

			return new Single(delay_value_pair);
		} else {
			int i = isFirstOrSecondType(str);
			if (i < 0) { // Type "long n (S)"
				int first_bracket = str.indexOf("(");
				AbstractScript feedback = subParse(str.substring(first_bracket+1,
						str.length() - 1));

				Pattern num_loop_pattern = Pattern.compile("[0-9]+");
				Matcher num_loop_matcher = num_loop_pattern.matcher(str);
				num_loop_matcher.find();
				String num_loop_str = num_loop_matcher.group();
				long num_loop = Long.parseLong(num_loop_str);

				return new Loop(num_loop, feedback);
			} else if (i > 0) { // Type "loop n (S) S"
				AbstractScript left = subParse(str.substring(0, i));
				AbstractScript right = subParse(str.substring(i));
				return new Sequence(left, right);
			} else { // Type "<long, LogicValue> S"
				int j = isThirdorFourthType(str);
				System.out.println("j="+j);
				System.out.println("Script under parsing is \""+str+"\"");
				assert j > 0;
				AbstractScript left = subParse(str.substring(0, j));
				AbstractScript right = subParse(str.substring(j));
				return new Sequence(left, right);
			}
		}
	}

	/**
	 * Get the logic value corresponding to l
	 * 
	 * @param str
	 * @return
	 */
	private static LogicValue getLogicValue(String str) {
		assert str.length() == 1;
		assert str.equals("0") || str.equals("1")
				|| str.toLowerCase().equals("x");
		if (str.equals("0")) {
			return LogicValue.ZERO;
		} else if (str.equals("1")) {
			return LogicValue.ONE;
		} else {
			return LogicValue.X;
		}
	}

	/**
	 * Test if the input string is of type "loop n (S)" or "loop n (S) S". This
	 * is done by counting brackets.
	 * 
	 * @param string
	 * @return i>0 - the input string is of type "loop n (S) S", i represents
	 *         the end index of "loop n (S)" + 1 -1 - the input string is of
	 *         type "loop n (S)" 0 - the input string is of other types
	 */
	private static int isFirstOrSecondType(String string) {
		string.trim();
		if (Pattern.matches("loop\\s+[0-9]+.+", string)) {
			char[] chars = string.toCharArray();
			int left_count = 0, right_count = 0;
			for (int i = 0; i < string.length(); i++) {
				if (left_count != 0 && left_count == right_count) {
					return i;
				}
				if (chars[i] == '(') {
					left_count++;
				} else if (chars[i] == ')') {
					right_count++;
				}
				assert left_count >= right_count;
			}
			assert left_count == right_count;
			return -1;
		}
		return 0;
	}

	/**
	 * Test if the input string is of type "<long, LogicValue>" or
	 * "<long, LogicValue> S". This is done by counting brackets.
	 * 
	 * @param string
	 * @return -1 - the input string is of type "<long, LogicValue>" i>0 - the
	 *         input string is of type "<long, LogicValue> S", i represents the
	 *         end index of "<long, LogicValue>" + 1
	 */
	private static int isThirdorFourthType(String string) {
		string.trim();
		Pattern single_pattern = Pattern
				.compile("<\\s*[0-9]+\\s*,\\s*[01xX]\\s*>");
		Matcher single_matcher = single_pattern.matcher(string);
		if (single_matcher.find() && single_matcher.start() == 0) {
			int end_index = single_matcher.end();
			if (end_index == string.length()) {
				return -1;
			} else {
				return end_index;
			}
		} else {
			return 0;
		}
	}

	// ====================================================
	// Inner classes
	// ====================================================

	static class Loop extends AbstractScript {
		private long n, m;
		private AbstractScript child;

		Loop(long n, AbstractScript child) {
			assert n > 0;
			this.n = n;
			this.m = n;
			this.child = child;
		}

		public long getTotalLoopCount() {
			return n;
		}

		public long getLoopCountDown() {
			return m;
		}

		public AbstractScript getChild() {
			return child;
		}

		@Override
		DelayValuePair getNextPair() {
			DelayValuePair temp = child.getNextPair();
			if (temp == null){
				m--;
				if (m<=0){
					return null;
				}else{
					child.restore();
					temp = child.getNextPair();
					assert temp != null;
					return temp;
				}
			}
			else{
				return temp;
			}
		}

		@Override
		public void restore() {
			this.m = n;
			child.restore();
		}

	}

	static class Sequence extends AbstractScript {
		private AbstractScript left, right;

		Sequence(AbstractScript left, AbstractScript right) {
			this.left = left;
			this.right = right;
		}

		@Override
		DelayValuePair getNextPair() {
			DelayValuePair temp = left.getNextPair();
			if (temp == null) {
				temp = right.getNextPair();
			}
			return temp;
		}

		@Override
		public void restore() {
			left.restore();
			right.restore();
		}

	}

	static class Single extends AbstractScript {
		private boolean visited;
		private DelayValuePair pair;

		Single(DelayValuePair pair) {
			this.pair = new DelayValuePair(pair);
		}

		public DelayValuePair getDelayValuePair() {
			return pair;
		}

		@Override
		DelayValuePair getNextPair() {
			if (visited)
				return null;
			else {
				visited = true;
				return pair;
			}
		}

		@Override
		public void restore() {
			this.visited = false;
		}
	}

	static class DelayValuePair {
		private long delay;
		private LogicValue value;

		DelayValuePair(long delay, LogicValue value) {
			this.delay = delay;
			this.value = value;
		}

		DelayValuePair(DelayValuePair pair) {
			this.delay = pair.getDelay();
			this.value = pair.getLogicValue();
		}

		public long getDelay() {
			return delay;
		}

		public LogicValue getLogicValue() {
			return value;
		}
		
		@Override
		public String toString(){
			return "<"+delay+","+value.toString()+">";
		}
	}
}
\end{lstlisting}

\begin{lstlisting}[caption=Scheduler, label=listing8]
public class Scheduler implements SchedulerInterface {
	private GateFactory gf;
	private TransitionEventFactory tef;
	private EventList el;
	private GateEventListManager gelm;
	private HashMap<String, String> most_recent_history;
	private static Scheduler instance;

	public static Scheduler getInstance() {
		if (instance == null) {
			instance = new Scheduler();
			instance.initialize();
		}
		return instance;
	}

	private Scheduler() {
		this.gf = GateFactory.getInstance();
		this.tef = TransitionEventFactory.getInstance();
		this.el = new EventList();
		this.gelm = new GateEventListManager();
		this.most_recent_history = new HashMap<String, String>();
	}

	protected void initialize() {
		gelm.initialize();
	}

	/* (non-Javadoc)
	 * @see com.asys.simulator.SchedulerInterface#getNextTransitionEventId()
	 */
	public String getNextTransitionEventId() throws IdNotExistException, SchedulerEmptyException {
		if (isEmpty()) {
			throw new SchedulerEmptyException();
		} else {
			String next_event_id = el.removeFirst();
			TransitionEvent next_event = tef.getTransitionEvent(next_event_id);
			EventList gel = gelm.getEventList(next_event.getGateId());
			String next_event_id2 = gel.removeFirst();
			assert next_event_id.equals(next_event_id2);
			return next_event_id;
		}
	}
	
	/* (non-Javadoc)
	 * @see com.asys.simulator.SchedulerInterface#getNextTransitionEventIdNoEffect()
	 */
	public String getNextTransitionEventIdNoEffect() throws IdNotExistException, SchedulerEmptyException{
		if (isEmpty()) {
			throw new SchedulerEmptyException();
		} else {
			String next_event_id = el.getFirst();
			TransitionEvent next_event = tef.getTransitionEvent(next_event_id);
			EventList gel = gelm.getEventList(next_event.getGateId());
			String next_event_id2 = gel.getFirst();
			assert next_event_id.equals(next_event_id2);
			return next_event_id;
		}
	}
	
	/* (non-Javadoc)
	 * @see com.asys.simulator.SchedulerInterface#reset()
	 */
	public void reset(){
		this.gf = GateFactory.getInstance();
		this.tef = TransitionEventFactory.getInstance();
		this.el = new EventList();
		this.gelm = new GateEventListManager();
		initialize();
	}

	/* (non-Javadoc)
	 * @see com.asys.simulator.SchedulerInterface#schedule(java.lang.String)
	 */
	public ArrayList<String> schedule(String transition_event_id)
			throws IdNotExistException, IdExistException {
		return scheduleEventNotForCGate(transition_event_id);
	}

	/**
	 * For non-C gates, their transition events are scheduled by the follow scheme:
	 * 
	 * If the event e1 to be scheduled is a start event, then all end events that are already scheduled 
	 * for the gate and are simultaneous with or later than e1 are removed.  If there is a start event
	 * immediately before e1, then e1 is STILL scheduled.  This is for checking common ambiguities.
	 * 
	 * If the event e2 to be scheduled is an end event, then it is added to the end of the events of 
	 * the gate.
	 * 
	 * There some important invariants about the list of scheduled transition events of a non-C gate:
	 * 1. There are not contiguous or simultaneous start events.
	 * 2. End events before start events must have been scheduled earlier than the start events.
	 * 3. Any newly start event scheduled must be the last start event in the queue.
	 * 4. Any newly end event scheduled must be the last end event in the queue.
	 * 
	 * @param transition_event_id
	 * @return an ArrayList of IDs of TransitionEvents that can be safely removed from transition event factory
	 * @throws IdNotExistException
	 * @throws IdExistException 
	 */
	private ArrayList<String> scheduleEventNotForCGate(
			String transition_event_id) throws IdNotExistException,
			IdExistException {
		// Assert that the transtion_event_id is NOT targeting a C element
		TransitionEvent event = tef.getTransitionEvent(transition_event_id);
		String gate_id = event.getGateId();
		// assert !(gate.getElement() instanceof CGate);

		ArrayList<String> waste = new ArrayList<String>();

		EventList gel = gelm.getEventList(gate_id);
		if (tef.isStartEvent(transition_event_id)) {
			/* 
			 * Remove all scheduled events for the gate after this start event.  
			 * By invariant 3, those removed can only be end events.
			 */
			List<String> useless_end_events = gel.removeEventsAtOrAfter(event
					.getCircuitTime());
			for (String id : useless_end_events) {
				waste.add(id);
			}
			el.removeEvents(useless_end_events);

			// Check if the previous event is a start event
			int i = gelm.getInsertionPosition(transition_event_id);
			String previous_event_id = gelm
					.getTransitionEventAt(gate_id, i - 1);
			if (previous_event_id != null
					&& tef.isStartEvent(previous_event_id)) {
				// The following is wrong: the event is not scheduled
				// This event is still scheduled
//				gel.insertEvent(transition_event_id);
//				el.insertEvent(transition_event_id);
			} else {// The previous event is not a start event
				gel.insertEvent(transition_event_id);
				el.insertEvent(transition_event_id);
			}
			return waste;
		} else {
			gel.insertEvent(transition_event_id);
			el.insertEvent(transition_event_id);
		}
		most_recent_history.put(gate_id, transition_event_id);
		return waste;
	}
	
	protected String getMostRecentHistory(String gate_id){
		return most_recent_history.get(gate_id);
	}

	protected EventList getEventListForGate(String gate_id) {
		if (gelm.hasKey(gate_id)) {
			try {
				return gelm.getEventList(gate_id);
			} catch (IdNotExistException e) {
				System.out.println("Should never have happened!");
				e.printStackTrace();
			}
			return null;
		} else {
			return null;
		}
	}

	protected int getTotalNumberOfEventsOnScheduler() {
		return el.getCurrentNumberOfEvents();
	}

	protected boolean isEmpty() {
		return el.getCurrentNumberOfEvents() <= 0;
	}

	// /**
	// * For C gates, their transition events are scheduled by the follow
	// scheme:
	// *
	// * To schedule a start event e1 on one input, firstly check the state of
	// the other input. If the
	// * other input has binary value, then
	// *
	// * @param transition_event_id
	// * @return
	// * @throws IdNotExistException
	// * @throws IdExistException
	// */
	// private ArrayList<String> scheduleEventForCGate(
	// String transition_event_id) throws IdNotExistException,
	// IdExistException {
	//
	//
	// }

	/**
	 * This class is for keeping track of the list of scheduled events.
	 * 
	 * @author ryan
	 * 
	 */
	public class EventList {
		private LinkedList<String> list;

		protected EventList() {
			list = new LinkedList<String>();
		}

		protected int getInsertionPosition(String event_id)
				throws IdExistException, IdNotExistException {
			if (list.contains(event_id)) {
				throw new IdExistException(event_id);
			} else {
				TransitionEvent new_event = tef.getTransitionEvent(event_id);
				assert new_event != null;
				int i = 0;
				for (String id : list) {
					TransitionEvent event = tef.getTransitionEvent(id);
					if (new_event.getCircuitTime() < event.getCircuitTime()
							|| (new_event.getCircuitTime() == event
									.getCircuitTime()
									&& tef.isStartEvent(event_id) && !tef
									.isStartEvent(id))) {
						break;
					}
					i++;
				}
				return i;
			}
		}

		/**
		 * Inserting an event_id into the list of event IDs.
		 * 
		 * The position of the inserted event_id depends on the time and type of the event inserted (start or end event).
		 * 
		 * The list is ordered by increasing circuit time.
		 * 
		 * If two events have the same circuit time, then start event has precedence over end event.
		 * 
		 * Otherwise, the first inserted event takes precedence.
		 * 
		 * @param event_id
		 * @return the index at which the new event ID is inserted.
		 * @throws IdExistException
		 * @throws IdNotExistException 
		 */
		protected int insertEvent(String event_id) throws IdExistException,
				IdNotExistException {
			int i = getInsertionPosition(event_id);
			list.add(i, event_id);
			return i;
		}

		protected String getFirst() {
			if (list.isEmpty()){
				return null;
			}else{
				return list.getFirst();
			}
		}

		protected String getLast() {
			if (list.isEmpty()){
				return null;
			}else{
				return list.getLast();
			}
		}

		protected String getAt(int i) {
			if (0 <= i && i < list.size()) {
				return list.get(i);
			} else {
				return null;
			}
		}

		protected String removeFirst() {
			return list.removeFirst();
		}

		protected String removeLast() {
			return list.removeLast();
		}

		protected boolean remove(String event_id) {
			return list.removeFirstOccurrence(event_id);
		}

		/**
		 * Return all event IDs stored.  It is important that the returned list is NOT modified in any way.
		 * @return
		 */
		protected LinkedList<String> getAllEvents() {
			return list;
		}

		/**
		 * Return an ArrayList of event IDs that are at or later than the time
		 * @param time
		 * @return
		 */
		protected List<String> getEventsAtOrAfter(long time) {
			int i = 0;
			for (String event_id : list) {
				TransitionEvent event = tef.getTransitionEvent(event_id);
				assert event != null;
				if (time <= event.getCircuitTime()) {
					break;
				}
				i++;
			}
			return list.subList(i, list.size());
		}

		/**
		 * Remove and return an ArrayList of event IDs that are at or later than the time
		 * @param time
		 * @return
		 */
		protected List<String> removeEventsAtOrAfter(long time) {
			int i = 0;
			for (String event_id : list) {
				TransitionEvent event = tef.getTransitionEvent(event_id);
				assert event != null;
				if (time <= event.getCircuitTime()) {
					break;
				}
				i++;
			}
			List<String> result = new LinkedList<String>();

			result.addAll(list.subList(i, list.size()));

			// Remove event IDs from the list
			int size = list.size();
			for (int j = i; j < size; j++) {
				list.remove(i);
			}

			return result;
		}

		protected void removeEvents(List<String> event_ids) {
			list.removeAll(event_ids);
		}

		protected int getCurrentNumberOfEvents() {
			return list.size();
		}
	}

	/**
	 * Used for tracking what events have been scheduled for each gate.
	 * @author ryan
	 * 
	 */
	private class GateEventListManager {
		private HashMap<String, EventList> map;

		protected GateEventListManager() {
			// Do nothing
		}

		protected void initialize() {
			this.map = new HashMap<String, EventList>();
			for (String gate_id : GateFactory.getInstance().getGateIds()) {
				map.put(gate_id, new EventList());
			}
		}

		protected boolean hasKey(String gate_id) {
			return map.containsKey(gate_id);
		}

		protected EventList getEventList(String gate_id)
				throws IdNotExistException {
			if (gf.gateIdExists(gate_id)) {
				return map.get(gate_id);
			} else {
				throw new IdNotExistException(gate_id);
			}
		}

		protected int getInsertionPosition(String transition_event_id)
				throws IdNotExistException, IdExistException {
			if (tef.transitionEventIdExists(transition_event_id)) {
				TransitionEvent transition_event = tef
						.getTransitionEvent(transition_event_id);
				String gate_id = transition_event.getGateId();
				return map.get(gate_id).getInsertionPosition(
						transition_event_id);
			} else {
				throw new IdNotExistException(transition_event_id);
			}
		}

		/**
		 * 
		 * @param transition_event_id
		 * @return the index at which the new event ID is inserted.
		 * @throws IdNotExistException
		 * @throws IdExistException
		 */
		protected int insertEvent(String transition_event_id)
				throws IdNotExistException, IdExistException {
			if (tef.transitionEventIdExists(transition_event_id)) {
				TransitionEvent transition_event = tef
						.getTransitionEvent(transition_event_id);
				String gate_id = transition_event.getGateId();
				return map.get(gate_id).insertEvent(transition_event_id);
			} else {
				throw new IdNotExistException(transition_event_id);
			}
		}

		protected String getTransitionEventAt(String gate_id, int i) {
			EventList list = map.get(gate_id);
			return list.getAt(i);
		}

		/**
		 * Return an ArrayList of event IDs that are at or later than the time
		 * @param gate_id
		 * @param time
		 * @return
		 */
		protected List<String> getTransitionEventsAtOrAfter(String gate_id,
				long time) {
			EventList list = map.get(gate_id);
			return list.getEventsAtOrAfter(time);
		}

		/**
		 * Remove and return an ArrayList of event IDs that are at or later than the time
		 * @param gate_id
		 * @param time
		 * @return
		 */
		protected List<String> removeTransitionEventsAfter(String gate_id,
				long time) {
			EventList list = map.get(gate_id);
			return list.removeEventsAtOrAfter(time);
		}

		protected void removeTransitionEvents(String gate_id,
				List<String> event_ids) {
			EventList list = map.get(gate_id);
			list.removeEvents(event_ids);
		}
	}
}
\end{lstlisting}





